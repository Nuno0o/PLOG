\documentclass[a4paper]{article}

%use the english line for english reports
%usepackage[english]{babel}
\usepackage[portuguese]{babel}
\usepackage[utf8]{inputenc}
\usepackage{indentfirst}
\usepackage{graphicx}
\usepackage{verbatim}


\begin{document}

\setlength{\textwidth}{16cm}
\setlength{\textheight}{22cm}

\title{\Huge\textbf{Título do Trabalho}\linebreak\linebreak\linebreak
\Large\textbf{Relatório Final}\linebreak\linebreak
\linebreak\linebreak
\includegraphics[scale=0.1]{feup-logo.png}\linebreak\linebreak
\linebreak\linebreak
\Large{Mestrado Integrado em Engenharia Informática e Computação} \linebreak\linebreak
\Large{Programação em Lógica}\linebreak
}

\author{\textbf{Grupo Ploy4:}\\ Gonçalo Ribeiro - 201403877 \\ Nuno Martins - 201405079 \\\linebreak\linebreak \\
 \\ Faculdade de Engenharia da Universidade do Porto \\ Rua Roberto Frias, s\/n, 4200-465 Porto, Portugal \linebreak\linebreak\linebreak
\linebreak\linebreak\vspace{1cm}}
\date{13 de Novembro de 2016}
\maketitle
\thispagestyle{empty}

%************************************************************************************************
%************************************************************************************************

\newpage

\section*{Resumo}
Para o 1º Trabalho Prático, o nosso grupo implementou o jogo de tabuleiro Ploy em Prolog - linguagem de programação em lógica.
O principal objectivo deste trabalho foi aplicar os conhecimentos adquiridos sobre o paradigma de programação em lógica, lecionados na unidade curricular PLOG.
O resultado final é um versão funcional do jogo com interface de texto. Foi também implementada um inteligência artificial capaz de fazer jogadas pseudoaleatórias.
Como considerações finais, estamos satisfeitos com o trabalho que realizámos e concordamos que foi uma experiência enriquecedora e que de certa forma colocou à prova os nossos conhecimentos de lógica e a nossa capacidade de aprender um novo paradigma de programação num espaço de tempo relativamente pequeno.

%Resumo sucinto do trabalho com 150 a 250 palavras (problema abordado, objetivo, como foi o problema resolvido/abordado, principais resultados e conclusões).

\newpage

\tableofcontents

%************************************************************************************************
%************************************************************************************************

%*************************************************************************************************
%************************************************************************************************

\newpage

%%%%%%%%%%%%%%%%%%%%%%%%%%
\section{Introdução}
No âmbito da unidade curricular de Programação em Lógica, foi-nos proposto para o primeiro trabalho prático a elaboração de um jogo de tabuleiro com a línguagem Prolog. O jogo foi selecionado de uma lista de jogos sedida pelos docentes. A escolha do Ploy como objecto do nosso trabalho residiu no interesse mútuo que temos por jogos da família do xadrez.
O objetivo deste trabalho foi pôr em prática os conhecimentos adquiridos sobre programação em lógica. Programação em lógica é um paradigma de programação baseado em lógica formal. Um programa escrito neste paragima é um conjunto de premissas e expressões em forma lógica, que expressam factos e regras sobre o dominio do problema. A línguagem Polog é uma das linguagens de programação lógica, desenvolvida por Alain Colmerauer - cientista de computação francês - em 1972.


%%%%%%%%%%%%%%%%%%%%%%%%%%
\section{O Jogo Ploy}

Descrever sucintamente o jogo, a sua história e, principalmente, as suas regras. Devem ser incluídas imagens apropriadas para explicar o funcionamento do jogo. (Pode ser idêntico ao texto do relatório intercalar.)


%%%%%%%%%%%%%%%%%%%%%%%%%%
\section{Lógica do Jogo}

Descrever o projeto e implementação da lógica do jogo em Prolog, incluindo a forma de representação do estado do tabuleiro e sua visualização, execução de movimentos, verificação do cumprimento das regras do jogo, determinação do final do jogo e cálculo das jogadas a realizar pelo computador utilizando diversos níveis de jogo. Sugere-se a estruturação desta secção da seguinte forma:

\subsection{Representação do Estado do Jogo} Pode ser idêntico ao descrito no relatório intercalar.)

\subsection{Visualização do Tabuleiro} (Pode ser idêntico ao descrito no relatório intercalar.)

\subsection{Lista de Jogadas Válidas} Obtenção de uma lista de jogadas possíveis. Exemplo: \textit{valid\_moves(+Board, -ListOfMoves)}.

\subsection{Execução de Jogadas} Validação e execução de uma jogada num tabuleiro, obtendo o novo estado do jogo. Exemplo: \textit{move(+Move, +Board, -NewBoard)}.

\subsection{Avaliação do Tabuleiro} Avaliação do estado do jogo, que permitirá comparar a aplicação das diversas jogadas disponíveis. Exemplo: \textit{value(+Board, +Player, -Value)}.

\subsection{Final do Jogo} Verificação do fim do jogo, com identificação do vencedor. Exemplo: \textit{game\_over(+Board, -Winner)}.

\subsection{Jogada do Computador} Escolha da jogada a efetuar pelo computador, dependendo do nível de dificuldade. Por exemplo: \textit{choose\_move(+Level, +Board, -Move)}.


%%%%%%%%%%%%%%%%%%%%%%%%%%
\section{Interface com o Utilizador}

Descrever o módulo de interface com o utilizador em modo de texto.


%%%%%%%%%%%%%%%%%%%%%%%%%%
\section{Conclusões}
Que conclui deste projecto? Como poderia melhorar o trabalho desenvolvido?


\clearpage
\addcontentsline{toc}{section}{Bibliografia}
\renewcommand\refname{Bibliografia}
\bibliographystyle{plain}
\bibliography{myrefs}

\newpage
\appendix
\section{Nome do Anexo}
Código Prolog implementado devidamente comentado e outros elementos úteis que não sejam essenciais ao relatório.

\end{document}
